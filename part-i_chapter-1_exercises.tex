\documentclass{article}

\usepackage{enumitem}
\setlist{nosep}  
\usepackage{amsmath}
\usepackage{setspace}
\usepackage{titlesec}
\titleformat*{\section}{\large\bfseries}

\title{Introduction to Logic, Part I, Chapter I by Patrick Suppes - exercises}
\author{Dominik Lenda}
\begin{document}
\maketitle

\section*{Exercise 4.}
In the following examples determine the truth value of the compoind sentences from the given truth values of the component sentences.
\medskip
\begin{enumerate}[label=(\roman*)]
    \item "Galileo was born before Descartes" is true.
    \item "Descartes was born in the sixteenth century" is true.
    \item "Newton was born before Shakespeare" is false.
    \item "Racine was a compatriot of Galileo" is false.
\end{enumerate}
\medskip
\begin{enumerate}[label=(\alph*)]
    \item If Galileo was born before Descartes, then Newton was not born before Shakespeare.\\
        Answer: $true \rightarrow \neg false \text{ is } true$
    \item If either Racine was a compatriot of Galileo or Newton was born before Shakespeare, then Descartes was born in the sixteenth century.
        Answer: $(false \vee false) \rightarrow true \text{ is } true$ 
    \item If Racine was not a compatriot of Galileo, then either Descartes was not born in the sixteenth century or Newton was born before Shakespeare.
        Answer: $\neg false \rightarrow (\neg true \vee false) \text{ is } false$
\end{enumerate}

\section*{Exercise 5.}
Let\\
$N = \text{New York is larger than Chicago}$\\
$W = \text{New York is north of Washington}$\\
$C = \text{Chicago is larger than New York}$\\
$N$, $W$ are true and $C$ is false.\\
Which of the following sentences are true?
\medskip
\begin{enumerate}[label=(\alph*)]
\item $N \vee C \text{ is } true$
\item $N \wedge C \text{ is } false$
\item $-N \wedge -C \text{ is } false$
\item $N \leftrightarrow -W \vee C \text{ is } false$
\item $W \vee -C \rightarrow N \text{ is } true$
\item $(W \vee N) \rightarrow (W \rightarrow -C) \text{ is } true$
\item $(W \leftrightarrow -N) \leftrightarrow (N \leftrightarrow C) \text{ is } true$
\item $(W \rightarrow N) \rightarrow [(N \rightarrow -C) \rightarrow (-C \rightarrow W)] \text{ is } true$
\end{enumerate}

\section*{Exercise 6.}
Let\\
$P = \text{Jane Austen was contemporary of Beethoven}$\\
$Q = \text{Beethoven was a contemporary of Gauss}$\\
$R = \text{Gauss was a contemporary of Napoleon}$\\
$S = \text{Napoleon was a contemporary of Julius Caesar}$\\
$P$, $Q$, and $R$ are true, and $S$ is false.\\
Find the truth values of the following sentences:
\medskip
\begin{enumerate}[label=(\alph*)]
\item $(P \wedge Q) \wedge R \text{ is } true$
\item $P \wedge (Q \wedge R) \text{ is } true$
\item $S \rightarrow P \text{ is } true$
\item $P \rightarrow S \text{ is } false$
\item $(P \wedge Q) \wedge (R \wedge S) \text{ is } false$
\item $P \wedge Q \leftrightarrow R \wedge -S \text{ is } true$
\item $(P \leftrightarrow Q) \rightarrow (S \leftrightarrow R) \text{ is } false$
\item $(-P \leftarrow Q) \leftarrow (S \leftarrow R) \text{ is } true$
\item $(P \rightarrow -Q) \rightarrow (S \leftrightarrow R) \text{ is } true$
\item $(P \rightarrow Q) [(Q \rightarrow R) \rightarrow (R \rightarrow S)] \text{ is } false$
\item $P \rightarrow [Q \leftrightarrow (R \rightarrow S)] \text{ is } false$
\end{enumerate}

\section*{Exercise 7.}
Let $P$ be a sentence such that for any sentence $Q$ the sentence $P \vee Q$ is true.\\
What can be said about the truth value of P.\\
Answer: $P \text{ is } true$

\section*{Exercise 8.}
Let $P$ be a sentence such that for any sentence $Q$ the sentence $P \wedge Q$ is false.\\
What can be said about the truth value of P.\\
Answer: $P \text{ is } false$

\section*{Exercise 9.} 
If $P \leftrightarrow Q$ is true, what can be said about the truth value of $P \vee -Q$?\\
Answer: \(P \vee -Q \text{ is } true\)

\section*{Exercise 10.}
\begin{enumerate}[label=(\alph*)]

\item \(P \lor Q\) is \textbf{not} a tautology.\\
\(
\begin{array}{|c c|c|}
p & q & p \lor q\\
\hline
T & T & T\\
T & F & T\\
F & T & T\\
F & F & F\\
\end{array}
\)
\medskip

\item \(P \lor -P\) is a tautology\\
\(
\begin{array}{|c|c|}
p & p \lor -p\\
\hline
T & T\\
F & T\\
\end{array}
\)
\medskip

\item \(P \lor Q \rightarrow Q \lor P\) is a tautology.\\
\(
\begin{array}{|c c|c|}
P & Q & P \lor Q \rightarrow Q \lor P\\
\hline
T & T & T\\
T & F & T\\
F & T & T\\
F & F & T\\
\end{array}
\)
\medskip

\item \(P \rightarrow (P \lor Q) \lor R\) is a tautology.\\
\(
\begin{array}{|c c c|c|}
P & Q & R & P \rightarrow (P \lor Q) \lor R\\
\hline
T & T & T & T\\
T & T & F & T\\
T & F & T & T\\
T & F & F & T\\
F & F & F & T\\
F & F & T & T\\
F & T & F & T\\
F & T & T & T\\
\end{array}
\)
\medskip

\item \(P \rightarrow (-P \rightarrow Q)\) is a tautology.\\
\(
\begin{array}{|c c|c|}
P & Q & P \rightarrow (-P \rightarrow Q)\\
\hline
T & T & T\\
T & F & T\\
F & T & T\\
F & F & T\\
\end{array}
\)
\medskip

\item \((P \rightarrow Q) \rightarrow (Q \rightarrow P)\) is \textbf{not} a tautology.\\
\(
\begin{array}{|c c|c|}
P & Q & P \rightarrow Q) \rightarrow (Q \rightarrow P\\
\hline
T & T & T\\
T & F & T\\
F & T & F\\
F & F & T\\
\end{array}
\)
\medskip

\item \([(P \rightarrow Q) \leftrightarrow Q] \rightarrow P\) is \textbf{not} a tautology.\\
\(
\begin{array}{|c c|c|}
P & Q &[(P \rightarrow Q) \leftrightarrow Q] \rightarrow P\\
\hline
T & T & T\\
T & F & T\\
F & T & F\\
F & F & T\\
\end{array}
\)
\medskip

\item \(P \rightarrow [Q \rightarrow (Q \rightarrow P)]\) is a tautology.\\
\(
\begin{array}{|c c|c|}
P & Q & P \rightarrow [Q \rightarrow (Q \rightarrow P)]\\
\hline
T & T & T\\
T & F & T\\
F & T & T\\
F & F & T\\
\end{array}
\)
\medskip

\item \(P \land Q \rightarrow P \lor R\) is a tautology.\\
\(
\begin{array}{|c c c|c|}
P & Q & R & P \land Q \rightarrow P \lor R\\
\hline
T & T & T & T\\
T & T & F & T\\
T & F & T & T\\
T & F & F & T\\
F & F & F & T\\
F & F & T & T\\
F & T & F & T\\
F & T & T & T\\
\end{array}
\)
\medskip

\item \([P \lor (-P \land Q)] \lor (-P \land -Q)\) is a tautology.\\
\(
\begin{array}{|c c|c|}
P & Q & [P \lor (-P \land Q)] \lor (-P \land -Q)\\
\hline
T & T & T\\
T & F & T\\
F & T & T\\
F & F & T\\
\end{array}
\)
\medskip

\item \(P \land Q \rightarrow (P \leftrightarrow Q \lor R) \) is a tautology.\\
\(
\begin{array}{|c c c|c|}
    P & Q & R & P \land Q \rightarrow (P \leftrightarrow Q \lor R)\\
\hline
T & T & T & T\\
T & T & F & T\\
T & F & T & T\\
T & F & F & T\\
F & F & F & T\\
F & F & T & T\\
F & T & F & T\\
F & T & T & T\\
\end{array}
\)
\medskip

\item \([P \land Q \rightarrow (P \land -P \rightarrow Q \lor -Q)] \land (Q \rightarrow Q)\) is a tautology.\\
\(
\begin{array}{|c c|c|}
P & Q & [P \land Q \rightarrow (P \land -P \rightarrow Q \lor -Q)] \land (Q \rightarrow Q)\\
\hline
T & T & T\\
T & F & T\\
F & T & T\\
F & F & T\\
\end{array}
\)
\medskip
\end{enumerate}

\section*{Exercise 11.}
If P and Q are distinct atomic sentences, which of the following are tautologies?
\singlespace
\begin{enumerate}[label=(\alph*)]

\item \(P \leftrightarrow Q\) is \textbf{not} a tautology.\\
\(
\begin{array}{|c c|c|}
P & Q & P \leftrightarrow Q\\
\hline
T & T & T\\
T & F & F\\
F & T & F\\
F & F & T\\
\end{array}
\)
\medskip

\item \(P \leftrightarrow P \lor P\) is a tautology.\\
\(
\begin{array}{|c|c|}
P  & P \leftrightarrow P \lor P\\
\hline
T & T\\
F & T\\
\end{array}
\)
\medskip

\item \(P \lor Q \leftrightarrow Q \lor P\) is a tautology.\\
\(
\begin{array}{|c c|c|}
P & Q & P \lor Q \leftrightarrow Q \lor P\\
\hline
T & T & T\\
T & F & T\\
F & T & T\\
F & F & T\\
\end{array}
\)
\medskip

\item \((P \rightarrow Q) \leftrightarrow (Q \rightarrow P)\) is \textbf{not} a tautology.\\
\(
\begin{array}{|c c|c|}
P & Q & (P \rightarrow Q) \leftrightarrow (Q \rightarrow P)\\
\hline
T & T & T\\
T & F & F\\
F & T & F\\
F & F & T\\
\end{array}
\)
\medskip

\item \((P \leftrightarrow P) \leftrightarrow P\) is a tautology.\\
\(
\begin{array}{|c|c|}
P  & (P \leftrightarrow P) \leftrightarrow P\\
\hline
T & T\\
F & T\\
\end{array}
\)
\medskip
\end{enumerate}

\end{document}
