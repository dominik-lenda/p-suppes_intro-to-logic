\documentclass{article}

\usepackage{enumitem}
\setlist{nosep}  
\usepackage{amsmath}
\usepackage{setspace}
\usepackage{titlesec}
\titleformat*{\section}{\large\bfseries}

\title{Introduction to Logic, P. Suppes - exercises}
\author{Dominik Lenda}

\begin{document}
\maketitle
\section*{Exercise 5.}
Let\\
$N = \text{New York is larger than Chicago}$\\
$W = \text{New York is north of Washington}$\\
$C = \text{Chicago is larger than New York}$\\
$N$, $W$ are true and $C$ is false.\\
Which of the following sentences are true?
\medskip
\begin{enumerate}[label=(\alph*)]
\item $N \vee C \text{ is } true$
\item $N \wedge C \text{ is } false$
\item $-N \wedge -C \text{ is } false$
\item $N \leftrightarrow -W \vee C \text{ is } false$
\item $W \vee -C \rightarrow N \text{ is } true$
\item $(W \vee N) \rightarrow (W \rightarrow -C) \text{ is } true$
\item $(W \leftrightarrow -N) \leftrightarrow (N \leftrightarrow C) \text{ is } true$
\item $(W \rightarrow N) \rightarrow [(N \rightarrow -C) \rightarrow (-C \rightarrow W)] \text{ is } true$
\end{enumerate}

\section*{Exercise 6.}
Let\\
$P = \text{Jane Austen was contemporary of Beethoven}$\\
$Q = \text{Beethoven was a contemporary of Gauss}$\\
$R = \text{Gauss was a contemporary of Napoleon}$\\
$S = \text{Napoleon was a contemporary of Julius Caesar}$\\
$P$, $Q$, and $R$ are true, and $S$ is false.\\
Find the truth values of the following sentences:
\medskip
\begin{enumerate}[label=(\alph*)]
\item $(P \wedge Q) \wedge R \text{ is } true$
\item $P \wedge (Q \wedge R) \text{ is } true$
\item $S \rightarrow P \text{ is } true$
\item $P \rightarrow S \text{ is } false$
\item $(P \wedge Q) \wedge (R \wedge S) \text{ is } false$
\item $P \wedge Q \leftrightarrow R \wedge -S \text{ is } true$
\item $(P \leftrightarrow Q) \rightarrow (S \leftrightarrow R) \text{ is } false$
\item $(-P \leftarrow Q) \leftarrow (S \leftarrow R) \text{ is } true$
\item $(P \rightarrow -Q) \rightarrow (S \leftrightarrow R) \text{ is } true$
\item $(P \rightarrow Q) [(Q \rightarrow R) \rightarrow (R \rightarrow S)] \text{ is } false$
\item $P \rightarrow [Q \leftrightarrow (R \rightarrow S)] \text{ is } false$
\end{enumerate}

\section*{Exercise 7.}
Let $P$ be a sentence such that for any sentence $Q$ the sentence $P \vee Q$ is true.\\
What can be said about the truth value of P.\\
Answer: $P \text{ is } true$

\section*{Exercise 8.}
Let $P$ be a sentence such that for any sentence $Q$ the sentence $P \wedge Q$ is false.\\
What can be said about the truth value of P.\\
Answer: $P \text{ is } false$

\section*{Exercise 9.} 
If $P \leftrightarrow Q$ is true, what can be said about the truth value of $P \vee -Q$?\\
Answer: $P \vee -Q \text{ is } true$
\end{document}
