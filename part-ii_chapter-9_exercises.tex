\documentclass{article}

\usepackage{enumitem}
\setlist{nosep}  
\usepackage{amsmath}
\usepackage{setspace}
\usepackage{titlesec}
\titleformat*{\section}{\large\bfseries}

\title{Introduction to Logic, Part II, Chapter 9 by Patrick Suppes - notes and exercises}
\author{Dominik Lenda}

\begin{document}
\maketitle

\section{Notes}
\emph{Principle of extensionality} - in axiomatic set theory\\
\smallskip
$\forall A \forall B (\forall X (X \in A \leftrightarrow X \in B) \rightarrow A = B)$\\
\smallskip
$\{1, 3, 5\} = \{5, 3, 1\}$\\
\smallskip
$\{1, 1, 3, 5\} = \{1, 3, 5\}$\\
\smallskip
$\{\text{Elizabeth II}\} \neq \text{Elizabeth II}$\\
Important difference between $A = A$ and $A \in A$ - the former is always true, while the latter is usually false. Standard systems of axiomatic set theory assert that set cannot be member of itself.\\
The relation of membership is not symmetric and transitive:
$2 \in \{1, 2\}$
$\{1, 2\} \notin 2$
\smallskip

\section*{Exercise 1.}
Which of the following statements are true (for all sets A, B, and C)?
\singlespace
\begin{enumerate}[label=(\alph*)]
\item If \(A = B\) and \(B = C\), then \(A = C\).
It is \emph{true} because the relation of identity is transitive.
\item If \(A \in B\) and \(B \in C\), then \(A \in C\).
It is \emph{false} because the relation of membership is \textbf{not} transitive.
\item If \(A \subseteq B\) and \(B \subseteq C\), then \(A \subseteq C\).
It is \emph{true} since the relation of inclusion is transitive.
\end{enumerate}
\end{document}
