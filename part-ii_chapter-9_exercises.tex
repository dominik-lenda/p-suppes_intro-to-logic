\documentclass{article}

\usepackage{enumitem}
\setlist{nosep}  
\usepackage{amsmath}
\usepackage{setspace}
\usepackage{titlesec}
\titleformat*{\section}{\large\bfseries}

\title{Introduction to Logic, Part II, Chapter 9 by Patrick Suppes - notes and exercises}
\author{Dominik Lenda}

\begin{document}
\maketitle

\section{Notes}
\emph{Principle of extensionality} - in axiomatic set theory\\
\smallskip
$\forall A \forall B (\forall X (X \in A \leftrightarrow X \in B) \rightarrow A = B)$\\
\smallskip
$\{1, 3, 5\} = \{5, 3, 1\}$\\
\smallskip
$\{1, 1, 3, 5\} = \{1, 3, 5\}$\\
\smallskip
$\{\text{Elizabeth II}\} \neq \text{Elizabeth II}$\\
Important difference between $A = A$ and $A \in A$ - the former is always true, while the latter is usually false. Standard systems of axiomatic set theory assert assert that set cannot be member of itself.
\smallskip
\end{document}
