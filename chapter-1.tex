\documentclass{article}

\usepackage{enumitem}
\setlist{nosep}  
\usepackage{amsmath}
\usepackage{xcolor}
\usepackage{setspace}
\usepackage{titlesec}
\titleformat*{\section}{\large\bfseries}

\title{Introduction to Logic, Part I, Chapter I by Patrick Suppes - exercises}
\author{Dominik Lenda}
\begin{document}
\maketitle

\section*{Exercise 1.}
A classical example of a non-truth-functional connective is that of possibility. For example, the sentence:

\begin{enumerate}
\item "It is possible that there is life on Mars."
\end{enumerate}
is true under any liberal interpretation of the notion of possibility; but then so is the sentence:
\begin{enumerate}[resume]
\item "It is possible that there is not any life on Mars."
\end{enumerate}
On the other hand, the sentence:
\begin{enumerate}[resume]
\item "It is possible that \(2 + 2 = 5\)"
\end{enumerate}
is ordinarly regarded as false. Using a exclamation mark "!" for "it is possible that", M for "there is life on Mars" and W for "2 + 2 = 5", we get the following tabular analysis of (1)-(3):

\[
\begin{array}{c|c|c|c|c|c}
M & -M & W & !M & !-M & !W\\
\hline
T & F & F & T & T & F\\
F & T &  & T & T & F\\
\end{array}
\]
\singlespace
The analysis of !M and !-M entails that the only truth-functiona lanalysis of the possibility connective is that for any sentence P, !P is true, but the truth value of !W controverts this; and we see that there is no appriopriate truth-functional analysis.
\singlespace
Give examples and an anlysis to show that the following are not truth-functional connectives:
\begin{enumerate}[label=(\alph*)]
    \item "Mr. Smith belives that carrots are beasts."
    \item "It is necessary that people eat pudding to stay fit."
\singlespace
\end{enumerate}

\textbf{This exercise is not finished. In my opinion it has too long introduction and vague goal, so I will not finish it.}

\section*{Exercise 2.}
Which of the truth-functional connectives introduced in this chapter is an approximate synonym of the connective "unless"? (Hint: To say "There will be peace unless there is a major war in the next five years, then there will be peace" is equicalent to saying "If there is not a major war in the next five years, then there will be peace".)
\singlespace
Answer: Negation of implication

\section*{Exercise 3.}

Translate the following compound sentences into symbolic notation, using letters to stand for atomic sentences. 

\begin{enumerate}[label=(\alph*)]
    \item "Either the fire was produced by arson or it was produced by spontaneous combustion."
\singlespace
    \(M =\) "the fire was produced by arson",\\
    \(L = \) "it was produced by spontaneous combustion"
\singlespace

\[M \lor L\]

    \item "If the water is clear, then either Henry can see the bottom of the pool or he is a nincompoop."
\singlespace
    \(M =\) "the water is clear",\\
    \(L = \) "Henry can see the bottom of the pool",\\
    \(S = \) "he is a nincompoop"
\singlespace

\[M \rightarrow L \lor S\]

    \item "Either John is not here or Mary is, and Helen certainly is."
\singlespace
    \(M =\) "John is here",\\
    \(L = \) "Mary is here",\\
    \(S = \) "Helen certainly is here"
\singlespace

\[-M \lor L \land S\]

    \item "If there are more cats than dogs, then there are more horses than dogs and there are fewer snakes than cats."
\singlespace
    \(M =\) "there are more cats than dogs",\\
    \(L = \) "there are more horses than dogs",\\
    \(S = \) "there are fewer snakes than cats"
\singlespace

\[M \rightarrow (L \land S)\]

    \item "The man in the moon is a fake, and if the same is true of Santa Claus, many children are deceived."
\singlespace
    \(M =\) "The man in the moon is a fake",\\
    \(L = \) "the same is true of Santa Claus",\\
    \(S = \) "many children are deceived"
\singlespace

\[M \land (L \rightarrow S)\]

    \item "If either red-heads are lovely or blondes do not have freckles, then logic is confusing."
\singlespace
    \(M =\) "red-heads are lovely",\\
    \(L = \) "blondes do not have freckles",\\
    \(S = \) "logic is confusing"
\singlespace

\[(M \lor L) \rightarrow S\]

    \item "If either housing is scarce or people like to live with their in-laws, and if people do not like to live with their in-laws, then housing is scarce."
\singlespace
    \(M =\) "housing is scarce",\\
    \(L = \) "people like to live with their in-laws",\\
\singlespace

\[M \lor L \land -L \rightarrow M\]


    \item "If John testifies and tells the truth, he will be found guilty; and if he does not testify, he will be found guilty."
\singlespace
    \(M =\) "John testifies",\\
    \(L = \) "John thells the truth",\\
    \(S = \) "he will be found guilty"
\singlespace

\[(M \land L \rightarrow S) \land (-M \rightarrow S)\]

    \item "Either John must testify and tell the truth, or he does not have to testify."
\singlespace
    \(M =\) "John must testify",\\
    \(L = \) "John must tell the truth",\\
\singlespace

\[(M \land L) \lor -M\]
\end{enumerate}

\section*{Exercise 4.}
In the following examples determine the truth value of the compound sentences from the given truth values of the component sentences.
\medskip
\begin{enumerate}[label=(\roman*)]
    \item "Galileo was born before Descartes" is true.
    \item "Descartes was born in the sixteenth century" is true.
    \item "Newton was born before Shakespeare" is false.
    \item "Racine was a compatriot of Galileo" is false.
\end{enumerate}
\medskip
\begin{enumerate}[label=(\alph*)]
    \item If Galileo was born before Descartes, then Newton was not born before Shakespeare.\\
        Answer: $true \rightarrow \neg false \text{ is } true$
    \item If either Racine was a compatriot of Galileo or Newton was born before Shakespeare, then Descartes was born in the sixteenth century.
        Answer: $(false \vee false) \rightarrow true \text{ is } true$ 
    \item If Racine was not a compatriot of Galileo, then either Descartes was not born in the sixteenth century or Newton was born before Shakespeare.
        Answer: $\neg false \rightarrow (\neg true \vee false) \text{ is } false$
\end{enumerate}

\section*{Exercise 5.}
Let\\
$N = \text{New York is larger than Chicago}$\\
$W = \text{New York is north of Washington}$\\
$C = \text{Chicago is larger than New York}$\\
$N$, $W$ are true and $C$ is false.\\
Which of the following sentences are true?
\medskip
\begin{enumerate}[label=(\alph*)]
\item $N \vee C \text{ is } true$
\item $N \wedge C \text{ is } false$
\item $-N \wedge -C \text{ is } false$
\item $N \leftrightarrow -W \vee C \text{ is } false$
\item $W \vee -C \rightarrow N \text{ is } true$
\item $(W \vee N) \rightarrow (W \rightarrow -C) \text{ is } true$
\item $(W \leftrightarrow -N) \leftrightarrow (N \leftrightarrow C) \text{ is } true$
\item $(W \rightarrow N) \rightarrow [(N \rightarrow -C) \rightarrow (-C \rightarrow W)] \text{ is } true$
\end{enumerate}

\section*{Exercise 6.}
Let\\
$P = \text{Jane Austen was contemporary of Beethoven}$\\
$Q = \text{Beethoven was a contemporary of Gauss}$\\
$R = \text{Gauss was a contemporary of Napoleon}$\\
$S = \text{Napoleon was a contemporary of Julius Caesar}$\\
$P$, $Q$, and $R$ are true, and $S$ is false.\\
Find the truth values of the following sentences:
\medskip
\begin{enumerate}[label=(\alph*)]
\item $(P \wedge Q) \wedge R \text{ is } true$
\item $P \wedge (Q \wedge R) \text{ is } true$
\item $S \rightarrow P \text{ is } true$
\item $P \rightarrow S \text{ is } false$
\item $(P \wedge Q) \wedge (R \wedge S) \text{ is } false$
\item $P \wedge Q \leftrightarrow R \wedge -S \text{ is } true$
\item $(P \leftrightarrow Q) \rightarrow (S \leftrightarrow R) \text{ is } false$
\item $(-P \leftarrow Q) \leftarrow (S \leftarrow R) \text{ is } true$
\item $(P \rightarrow -Q) \rightarrow (S \leftrightarrow R) \text{ is } true$
\item $(P \rightarrow Q) [(Q \rightarrow R) \rightarrow (R \rightarrow S)] \text{ is } false$
\item $P \rightarrow [Q \leftrightarrow (R \rightarrow S)] \text{ is } false$
\end{enumerate}

\section*{Exercise 7.}
Let $P$ be a sentence such that for any sentence $Q$ the sentence $P \vee Q$ is true.\\
What can be said about the truth value of P.\\
Answer: $P \text{ is } true$

\section*{Exercise 8.}
Let $P$ be a sentence such that for any sentence $Q$ the sentence $P \wedge Q$ is false.\\
What can be said about the truth value of P.\\
Answer: $P \text{ is } false$

\section*{Exercise 9.} 
If $P \leftrightarrow Q$ is true, what can be said about the truth value of $P \vee -Q$?\\
Answer: \(P \vee -Q \text{ is } true\)

\section*{Exercise 10.}
\begin{enumerate}[label=(\alph*)]

\item \(P \lor Q\) is \textbf{not} a tautology.\\
\(
\begin{array}{|c c|c|}
p & q & p \lor q\\
\hline
T & T & T\\
T & F & T\\
F & T & T\\
F & F & F\\
\end{array}
\)
\medskip

\item \(P \lor -P\) is a tautology\\
\(
\begin{array}{|c|c|}
p & p \lor -p\\
\hline
T & T\\
F & T\\
\end{array}
\)
\medskip

\item \(P \lor Q \rightarrow Q \lor P\) is a tautology.\\
\(
\begin{array}{|c c|c|}
P & Q & P \lor Q \rightarrow Q \lor P\\
\hline
T & T & T\\
T & F & T\\
F & T & T\\
F & F & T\\
\end{array}
\)
\medskip

\item \(P \rightarrow (P \lor Q) \lor R\) is a tautology.\\
\(
\begin{array}{|c c c|c|}
P & Q & R & P \rightarrow (P \lor Q) \lor R\\
\hline
T & T & T & T\\
T & T & F & T\\
T & F & T & T\\
T & F & F & T\\
F & F & F & T\\
F & F & T & T\\
F & T & F & T\\
F & T & T & T\\
\end{array}
\)
\medskip

\item \(P \rightarrow (-P \rightarrow Q)\) is a tautology.\\
\(
\begin{array}{|c c|c|}
P & Q & P \rightarrow (-P \rightarrow Q)\\
\hline
T & T & T\\
T & F & T\\
F & T & T\\
F & F & T\\
\end{array}
\)
\medskip

\item \((P \rightarrow Q) \rightarrow (Q \rightarrow P)\) is \textbf{not} a tautology.\\
\(
\begin{array}{|c c|c|}
P & Q & P \rightarrow Q) \rightarrow (Q \rightarrow P\\
\hline
T & T & T\\
T & F & T\\
F & T & F\\
F & F & T\\
\end{array}
\)
\medskip

\item \([(P \rightarrow Q) \leftrightarrow Q] \rightarrow P\) is \textbf{not} a tautology.\\
\(
\begin{array}{|c c|c|}
P & Q &[(P \rightarrow Q) \leftrightarrow Q] \rightarrow P\\
\hline
T & T & T\\
T & F & T\\
F & T & F\\
F & F & T\\
\end{array}
\)
\medskip

\item \(P \rightarrow [Q \rightarrow (Q \rightarrow P)]\) is a tautology.\\
\(
\begin{array}{|c c|c|}
P & Q & P \rightarrow [Q \rightarrow (Q \rightarrow P)]\\
\hline
T & T & T\\
T & F & T\\
F & T & T\\
F & F & T\\
\end{array}
\)
\medskip

\item \(P \land Q \rightarrow P \lor R\) is a tautology.\\
\(
\begin{array}{|c c c|c|}
P & Q & R & P \land Q \rightarrow P \lor R\\
\hline
T & T & T & T\\
T & T & F & T\\
T & F & T & T\\
T & F & F & T\\
F & F & F & T\\
F & F & T & T\\
F & T & F & T\\
F & T & T & T\\
\end{array}
\)
\medskip

\item \([P \lor (-P \land Q)] \lor (-P \land -Q)\) is a tautology.\\
\(
\begin{array}{|c c|c|}
P & Q & [P \lor (-P \land Q)] \lor (-P \land -Q)\\
\hline
T & T & T\\
T & F & T\\
F & T & T\\
F & F & T\\
\end{array}
\)
\medskip

\item \(P \land Q \rightarrow (P \leftrightarrow Q \lor R) \) is a tautology.\\
\(
\begin{array}{|c c c|c|}
    P & Q & R & P \land Q \rightarrow (P \leftrightarrow Q \lor R)\\
\hline
T & T & T & T\\
T & T & F & T\\
T & F & T & T\\
T & F & F & T\\
F & F & F & T\\
F & F & T & T\\
F & T & F & T\\
F & T & T & T\\
\end{array}
\)
\medskip

\item \([P \land Q \rightarrow (P \land -P \rightarrow Q \lor -Q)] \land (Q \rightarrow Q)\) is a tautology.\\
\(
\begin{array}{|c c|c|}
P & Q & [P \land Q \rightarrow (P \land -P \rightarrow Q \lor -Q)] \land (Q \rightarrow Q)\\
\hline
T & T & T\\
T & F & T\\
F & T & T\\
F & F & T\\
\end{array}
\)
\medskip
\end{enumerate}

\section*{Exercise 11.}
If P and Q are distinct atomic sentences, which of the following are tautologies?
\singlespace
\begin{enumerate}[label=(\alph*)]

\item \(P \leftrightarrow Q\) is \textbf{not} a tautology.\\
\(
\begin{array}{|c c|c|}
P & Q & P \leftrightarrow Q\\
\hline
T & T & T\\
T & F & F\\
F & T & F\\
F & F & T\\
\end{array}
\)
\medskip

\item \(P \leftrightarrow P \lor P\) is a tautology.\\
\(
\begin{array}{|c|c|}
P  & P \leftrightarrow P \lor P\\
\hline
T & T\\
F & T\\
\end{array}
\)
\medskip

\item \(P \lor Q \leftrightarrow Q \lor P\) is a tautology.\\
\(
\begin{array}{|c c|c|}
P & Q & P \lor Q \leftrightarrow Q \lor P\\
\hline
T & T & T\\
T & F & T\\
F & T & T\\
F & F & T\\
\end{array}
\)
\medskip

\item \((P \rightarrow Q) \leftrightarrow (Q \rightarrow P)\) is \textbf{not} a tautology.\\
\(
\begin{array}{|c c|c|}
P & Q & (P \rightarrow Q) \leftrightarrow (Q \rightarrow P)\\
\hline
T & T & T\\
T & F & F\\
F & T & F\\
F & F & T\\
\end{array}
\)
\medskip

\item \((P \leftrightarrow P) \leftrightarrow P\) is a tautology.\\
\(
\begin{array}{|c|c|}
P  & (P \leftrightarrow P) \leftrightarrow P\\
\hline
T & T\\
F & T\\
\end{array}
\)
\medskip
\end{enumerate}

\section*{Exercise 12.}
On the basis of ordinary usage construct truth tables for the sentential connectives used in the following examples:
\singlespace
\begin{enumerate}[label=(\alph*)]
\item Not both P and Q.\\
\(
\begin{array}{|c c|c|}
P & Q & -(P \land Q)\\ 
\hline
T & T & F\\
T & F & T\\
F & T & T\\
F & F & T\\
\end{array}
\)
\medskip
    
\item Neither P nor Q.\\
\(
\begin{array}{|c c|c|}
P & Q & -(P \lor Q)\\ 
\hline
T & T & F\\
T & F & F\\
F & T & F\\
F & F & T\\

\end{array}
\)
\medskip
\end{enumerate}

\section*{Exercise 13.}
Give examples of sentences P and Q (not necessarily atomic) such that the following compound sentences are tautologies.
\singlespace
\begin{enumerate}[label=(\alph*)]
\item Sentence \(W = P \land Q\) is \textbf{not} a tautology. Assumption \(P = P \lor -P\) and \(Q = Q \lor -Q\) changes \(W\) into a tautology.\\
\(
\begin{array}{|c c|c|c|}
P & Q & P \land Q & (P \lor -P) \land (Q \lor -Q)\\
\hline
T & T & T & T\\
T & F & F & T\\
F & T & F & T\\
F & F & F & T\\
\end{array}
\)
\medskip
\item Sentence \(W = P \lor (P \land -Q)\) is \textbf{not} a tautology. Assumption \(P = P \lor -P\) changes \(W\) into a tautology.\\ 
\(
\begin{array}{|c c|c|c|}
P & Q & P \lor (P \land -Q) & (P \lor -P) \lor (Q \land -Q)\\
\hline
T & T & T & T\\
T & F & T & T\\
F & T & F & T\\
F & F & F & T\\
\end{array}
\)
\medskip

\item Sentence \(W = P \rightarrow P \land -Q\) is \textbf{not} a tautology. Assumption \(Q = -P\) changes \(W\) into a tautology.\\ 
\(
\begin{array}{|c c|c|c|}
    P & Q & P \rightarrow P \land -Q & P \rightarrow P \land -(-P)\\
\hline
T & T & F & T\\
T & F & T & T\\
F & T & T & T\\
F & F & T & T\\
\end{array}
\)
\medskip

\item Sentence \(W = P \rightarrow -P\) is \textbf{not} a tautology. Assumption \(P = -(P \lor -P)\) changes \(W\) into a tautology.\\ 
\(
\begin{array}{|c|c|c|}
    P & P \rightarrow -P & -(P \lor -P) \rightarrow -(-(P \lor -P))\\
\hline
T & F & T\\
F & T & T\\
\end{array}
\)
\medskip
\end{enumerate}

\section*{Exercise 14.}
Is there any sentence \(P\) such that \(P \land -P\) is a tautology?\\
Answer: No such sentence exists.

\section*{Exercise 15.}
If \(P\) and \(Q\) are distinct atomic sentences, the sentence \(P \land Q\) tautologically implies which of the following?
\singlespace

\begin{enumerate}[label=(\alph*)]
\item \(P\)\\
\(
\begin{array}{|c c|c|}
P & Q & P \land Q \rightarrow P\\ 
\hline
T & T & T\\
T & F & T\\
F & T & T\\
F & F & T\\
\end{array}
\)
\singlespace
Answer: \(P \land Q \) tautologically implies \(P\).\\
\medskip

\item \(Q\)\\
\(
\begin{array}{|c c|c|}
P & Q & P \land Q \rightarrow Q\\ 
\hline
T & T & T\\
T & F & T\\
F & T & T\\
F & F & T\\
\end{array}
\)
\singlespace
Answer: \(P \land Q \) tautologically implies \(Q\).\\
\medskip

\item \(P \lor Q\)\\
\(
\begin{array}{|c c|c|}
P & Q & P \land Q \rightarrow P \lor Q\\ 
\hline
T & T & T\\
T & F & T\\
F & T & T\\
F & F & T\\
\end{array}
\)
\singlespace
Answer: \(P \land Q \) tautologically implies \(P \lor Q\).\\
\medskip

\item \(P \land -Q\)\\
\(
\begin{array}{|c c|c|}
P & Q & P \land Q \rightarrow P \land -Q\\ 
\hline
T & T & F\\
T & F & T\\
F & T & T\\
F & F & T\\
\end{array}
\)
\singlespace
Answer: \(P \land Q \) \textbf{does not} tautologically imply \(P \land -Q\).\\
\medskip

\item \(-P \lor Q\)\\
\(
\begin{array}{|c c|c|}
P & Q & P \land Q \rightarrow -P \lor Q\\ 
\hline
T & T & T\\
T & F & T\\
F & T & T\\
F & F & T\\
\end{array}
\)
\singlespace
Answer: \(P \land Q \) tautologically implies \(-P \lor Q\).\\
\medskip

\item \(-Q \rightarrow P\)\\
\(
\begin{array}{|c c|c|}
P & Q & P \land Q \rightarrow (-Q \rightarrow P)\\ 
\hline
T & T & T\\
T & F & T\\
F & T & T\\
F & F & T\\
\end{array}
\)
\singlespace
Answer: \(P \land Q \) tautologically implies \(-Q \rightarrow P\).\\
\medskip

\item \(P \leftrightarrow Q\)\\
\(
\begin{array}{|c c|c|}
P & Q & P \land Q \rightarrow (P \leftrightarrow Q)\\ 
\hline
T & T & T\\
T & F & T\\
F & T & T\\
F & F & T\\
\end{array}
\)
\singlespace
Answer: \(P \land Q \) tautologically implies \(P \leftrightarrow Q\).\\
\medskip
\end{enumerate}

\section*{Exercise 16.}
If \(P\) and \(Q\) are distinct atomic sentences, the sentence \(-P \lor Q\) tautologically implies which of the following?
\singlespace

\begin{enumerate}[label=(\alph*)]
\item \(P\)\\
\(
\begin{array}{|c c|c|}
P & Q & -P \lor Q \rightarrow P\\ 
\hline
T & T & T\\
T & F & T\\
F & T & F\\
F & F & F\\
\end{array}
\)
\singlespace
Answer: \(-P \lor Q\) \textbf{does not} tautologically imply \(P\).\\
\medskip

\item \(Q \rightarrow P\)\\
\(
\begin{array}{|c c|c|}
P & Q & -P \lor Q \rightarrow (Q \rightarrow P)\\ 
\hline
T & T & T\\
T & F & T\\
F & T & F\\
F & F & T\\
\end{array}
\)
\singlespace
Answer: \(-P \lor Q\) \textbf{does not} tautologically imply \(Q \rightarrow P\).\\
\medskip

\item \(P \rightarrow Q\)\\
\(
\begin{array}{|c c|c|}
P & Q & -P \lor Q \rightarrow (P \rightarrow Q)\\ 
\hline
T & T & T\\
T & F & T\\
F & T & T\\
F & F & T\\
\end{array}
\)
\singlespace
Answer: \(-P \lor Q\) tautologically implies \(P \rightarrow Q\).\\
\medskip

\item \(-Q \rightarrow -P\)\\
\(
\begin{array}{|c c|c|}
P & Q & -P \lor Q \rightarrow (-Q \rightarrow -P)\\ 
\hline
T & T & T\\
T & F & T\\
F & T & T\\
F & F & T\\
\end{array}
\)
\singlespace
Answer: \(-P \lor Q\) tautologically implies \(-Q \rightarrow -P\).\\
\medskip

\item \(-P \land Q\)\\
\(
\begin{array}{|c c|c|}
P & Q & -P \lor Q \rightarrow -P \land Q\\ 
\hline
T & T & F\\
T & F & T\\
F & T & T\\
F & F & F\\
\end{array}
\)
\singlespace
Answer: \(-P \lor Q\) \textbf{does not} tautologically imply \(-P \land Q\).\\
\medskip

\end{enumerate}

\section*{Exercise 17.}
If \(P\) and \(Q\) are distinct atomic sentences, the sentence \(P\) is tautologically equivalent to which of the following?
\begin{enumerate}[label=(\alph*)]

\item \(P \lor Q\)\\
\(
\begin{array}{|c c|c|}
P & Q & P \leftrightarrow P \lor Q\\ 
\hline
T & T & T\\
T & F & T\\
F & T & F\\
F & F & T\\
\end{array}
\)
\singlespace
Answer: \(P\) is \textbf{not} tautologically equivalent to \(P \lor Q\).\\
\medskip

\item \(P \lor -P\)\\
\(
\begin{array}{|c|c|}
P & P \leftrightarrow P \lor -P\\ 
\hline
T & T\\
F & F\\
\end{array}
\)
\singlespace
Answer: \(P\) is \textbf{not} tautologically equivalent to \(P \lor -P\).\\
\medskip

\item \(P \land P\)\\
\(
\begin{array}{|c|c|}
P & P \leftrightarrow P \land P\\ 
\hline
T & T\\
F & T\\
\end{array}
\)
\singlespace
Answer: \(P\) is tautologically equivalent to \(P \land P\).\\
\medskip

\item \(P \rightarrow P\)\\
\(
\begin{array}{|c|c|}
P & P \leftrightarrow P \rightarrow P\\ 
\hline
T & T\\
F & F\\
\end{array}
\)
\singlespace
Answer: \(P\) is \textbf{not} tautologically equivalent to \(P \rightarrow P\).\\
\medskip

\item \(-P \rightarrow P\)\\
\(
\begin{array}{|c|c|}
P & P \leftrightarrow -P \rightarrow P\\ 
\hline
T & T\\
F & T\\
\end{array}
\)
\singlespace
Answer: \(P\) is  tautologically equivalent to \(-P \rightarrow P\).\\
\medskip

\item \(P \rightarrow -P\)\\
\(
\begin{array}{|c|c|}
P & P \leftrightarrow P \rightarrow -P\\ 
\hline
T & F\\
F & F\\
\end{array}
\)
\singlespace
Answer: \(P\) is \textbf{not} tautologically equivalent to \(P \rightarrow -P\).\\
\medskip

\item \(Q \lor -Q \rightarrow P\)\\
\(
\begin{array}{|c c|c|}
P & Q & P \leftrightarrow (Q \lor -Q \rightarrow P)\\ 
\hline
T & T & T\\
T & F & T\\
F & T & T\\
F & F & T\\
\end{array}
\)
\singlespace
Answer: \(P\) is tautologically equivalent to \(Q \lor -Q \rightarrow P\).\\
\medskip
\end{enumerate}

\end{document}
